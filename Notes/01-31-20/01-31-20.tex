\documentclass[12pt]{article}

\usepackage[margin=1in]{geometry}
\usepackage{amsthm}
\usepackage{amssymb}
\usepackage{fullpage} % Package to use full page
\usepackage{parskip} % Package to tweak paragraph skipping
\usepackage{tikz} % Package for drawing
\usepackage{amsmath}
\usepackage{hyperref}
\usepackage{graphicx}

\begin{document}

\title{01-03-20}
\author{}
\maketitle

\section{Dust Corrections}
To correct for dust, we need an understanding of:
\begin{itemize}
    \item Where dust is
    \item What the dust is made of
    \item How much dust there is
\end{itemize}

A dust map like Schlegel+(1998) will give you a quantity $E(B-V)$ at the RA and dec that you input, where
\begin{align}
    E(B-V) = (B - V)_{\text{observed}} - (B - V)_{\text{intrinsic}}.
\end{align}
In other words, the observed $(B-V)$ color of a source will be reddened by some value $E(B-V)$, which is a function of how much dust is along the line of sight.

If we want to adjust an entire spectrum, we can make use of the total extinction (which is strongly related to the chemical composition of the dust)
\begin{align}
    A_\lambda = -2.5 \log \frac{F_\lambda}{F_{\lambda,0}}
\end{align}
where $F_{\lambda}$ is the extincted flux at some wavelength, and $F_{\lambda,0}$ is the unextincted flux at that wavelength.

The connection between the total extinction and the reddening is given as:
\begin{align}
    R = \frac{A_V}{E(B-V)},
\end{align}
where $R = 3.1$ is the Cardelli+(1989) extinction law found for the dust composition in the Milky Way.
\end{document}
