\documentclass[12pt]{article}

\usepackage[margin=1in]{geometry}
\usepackage{amsthm}
\usepackage{amssymb}
\usepackage{fullpage} % Package to use full page
\usepackage{parskip} % Package to tweak paragraph skipping
\usepackage{tikz} % Package for drawing
\usepackage{amsmath}
\usepackage{hyperref}
\usepackage{graphicx}

\begin{document}

\title{01-03-20}
\author{}
\maketitle

\section{Continuum Fitting Addendum}
In order to fit the continuum, we should fit a straight line through the fitting windows \textbf{in log-log space}. This is equivalent to fitting a power law in linear space. 

If we start with a power law in linear space:
\begin{align*}
    y = A x^m,
\end{align*}
then
\begin{align*}
    \log y = m \log x + \log A,
\end{align*}
so 
\begin{align*}
    \log y = m \log x + B,
\end{align*}
which is a straight line in log-log space.

It turns out the iron template fitting is easier in linear space, so we'll need to use something other than numpy.polyfit to do the power-law fitting. The best alternative is probably scipy.optimize.curve\_fit.
\section{Continuum + Iron Fitting}
    Instead of doing just continuum subtraction, we will fit the underlying continuum and iron emission at the same time, then subtract both from the spectrum.

    Before fitting:
    \begin{itemize}
        \item Interpolate the iron template.
        \item Code a basic function for a Gaussian function, $g(x, \mu, \sigma)$, where $\mu$ will be the center of the iron emission, and $\sigma$ will be the $\sigma_{conv}$ from the paper.
        \item Code a 3-parameter function that returns the convolution of the Gaussian broadening function and the $\log$-spaced template, $f(\lambda) \star g(\lambda, \mu, \sigma)$
    \end{itemize}

    Now the fitting (CIV for example):
    \begin{itemize}
        \item "Rebin" your data into log-spacing.
        \item Gather the flux values in the regions blue-ward [1435 - 1465\AA] and red-ward [1690 - 1710\AA] of the CIV line.
        \item Fit a line of the following form:
    \end{itemize}
    \begin{align*}
        F_{cont+iron}(A, k, B, \mu, \sigma) = A \lambda^k + B F_{iron}(\lambda, \mu, \sigma)
    \end{align*}
    \begin{itemize}
        \item Subtract this new fit from the spectrum.
    \end{itemize}
\end{document}
