
\documentclass[12pt]{article}

\usepackage[margin=1in]{geometry}
\usepackage{amsthm}
\usepackage{amssymb}
\usepackage{fullpage} % Package to use full page
\usepackage{parskip} % Package to tweak paragraph skipping
\usepackage{tikz} % Package for drawing
\usepackage{amsmath}
\usepackage{hyperref}
\usepackage{graphicx}

\begin{document}

\title{10-16-19}
\author{}
\maketitle
    \section{Procedure}
        \begin{itemize}
            \item Gather SDSS quasar spectra
            \item Include iron line templates
            \item Dust correct
            \item Continuum subtract
            \item Fit emission line profiles
            \item Analyze variability of CIV relative to MgII
            \item Identify spectra with repeat-observations
            \item Determine effects on black hole masses
        \end{itemize}

    \section{How to Continuum Subtract}
        See $\S3.1$ in \url{https://ui.adsabs.harvard.edu/abs/2016ApJS..224...14D/abstract}
        and $\S3.3$ in \url{https://ui.adsabs.harvard.edu/abs/2011ApJS..194...45S/abstract}

        We have \textit{continuum} emission below the CIV line that comes from the combined light of the galaxy + AGN.
        To properly isolate CIV (and MgII) emission and analyze line properties, we have to account for this excess emission.

        When subtracting continuum in this way, we have to be careful to keep track of uncertainties that propagate into our final results.
        When curve-fitting, the reliability of your fit depends heavily on the errorbars of the data, i.e, if all of your data has high variance, then your fit will be unreliable.
        Further, data with small errorbars should play into the fit more than data with large errorbars, since data with small errorbars is more reliable.
        Hence we should weight our fit in such a way that small variances are weighted more highly than large variances.
        A good rule of thumb is, for each data point $(x_i, y_i)$ we can define a weight such that:
        \begin{align}
            w_i \propto \frac{1}{\sigma_i^2},
        \end{align}
        where $\sigma_i$ is variance at each data point.

        The continuum subtraction itself should be done separately for MgII and CIV.

        For CIV:
        \begin{itemize}
            \item Calculate the mean flux in the regions blue-ward $[1435 - 1465 \AA]$ and red-ward $[1690 - 1710 \AA]$ of the CIV line.
            \item Gather up the variances associated with each spectral element (check spectrum data model for inverse variance).
            \item Fit a straight line through these two windows (accounting for variance).
            \item Subtract the straight line fit from the spectrum.
        \end{itemize}

        For MgII:
        \begin{itemize}
            \item Calculate the mean flux in the regions blue-ward $[2200 - 2700 \AA]$ and red-ward $[2900 - 3090 \AA]$ of the MgII line.
            \item Gather up the variances associated with each spectral element (check spectrum data model for inverse variance).
            \item Fit a straight line through these two windows (accounting for variance).
            \item Subtract the straight line fit from the spectrum.
        \end{itemize}

\end{document}
